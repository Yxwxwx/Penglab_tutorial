\documentclass[12pt, a4paper, oneside]{ctexart}
\usepackage{amsmath, amsthm, amssymb, appendix, bm, graphicx, hyperref, mathrsfs, xcolor}
\usepackage{listings}
\usepackage{ctex}
% 用来设置附录中代码的样式

\lstset{
    basicstyle          =   \sffamily,          % 基本代码风格
    keywordstyle        =   \bfseries,          % 关键字风格
    commentstyle        =   \rmfamily\itshape,  % 注释的风格,斜体
    stringstyle         =   \ttfamily,  % 字符串风格
    flexiblecolumns,                % 别问为什么,加上这个
    numbers             =   left,   % 行号的位置在左边
    showspaces          =   false,  % 是否显示空格,显示了有点乱,所以不现实了
    numberstyle         =   \zihao{-5}\ttfamily,    % 行号的样式,小五号,tt等宽字体
    showstringspaces    =   false,
    captionpos          =   t,      % 这段代码的名字所呈现的位置,t指的是top上面
    frame               =   lrtb,   % 显示边框
}

\lstdefinestyle{Python}{
    language        =   Python, % 语言选Python
    basicstyle      =   \zihao{-5}\ttfamily,
    numberstyle     =   \zihao{-5}\ttfamily,
    keywordstyle    =   \color{blue},
    keywordstyle    =   [2] \color{teal},
    stringstyle     =   \color{magenta},
    commentstyle    =   \color{red}\ttfamily,
    breaklines      =   true,   % 自动换行,建议不要写太长的行
    columns         =   fixed,  % 如果不加这一句,字间距就不固定,很丑,必须加
    basewidth       =   0.5em,
}

\title{\textbf{多参考方法中旋轨耦合矩阵元(SOCME)与系间窜越速率(ISC)的计算}}
\author{Sun Xinyu\\sunxinyu347@gmail.com}
\date{\today}
\linespread{1}


\begin{document}

\maketitle

\setcounter{page}{0}
\maketitle
\thispagestyle{empty}

\newpage
\pagenumbering{Roman}
\setcounter{page}{1}
\tableofcontents
\newpage
\setcounter{page}{1}
\pagenumbering{arabic}

\section{前言}

  系间窜越速率(ISC)常被用于分子发光体系的计算,最常用的方法是TD-DFT,这是因为其计算量小,对大多数有机发光体系支持都很好。其理论基础是Marcus理论,所以也可用于与电子转移速率相关的计算。然而,对于一般的态-态转换反应,单Slater行列式无法准确描述,故使用多参考方法计算。本文以铁配合物三-五重态反应为例,介绍使用多参考方法计算态态转换反应中SOCME和ISC\par
这是南开大学彭谦课题组新人入组手册系列之一,gitlab地址为$https://github.com/Yxwxwx/Penglab\_tutorial$
\newpage

\section{旋轨耦合理论基础}

  注意,大家在中级无机化学课程上已经接触过旋轨耦合效应这个概念,用来解释d-d跃迁等。而旋轨耦合效应是Dirac方程的解,必须通过相对论量子力学来解释。\par
  这里不写公式推导,只摆出来结论。

\subsection{Dirac方程}

  狄拉克方程是薛定谔方程和狭义相对论的结合,其数学形式为:
\begin{equation}
[c(\bm{\alpha}\cdot\mathbf{c})+m_0c^2\bm{\beta}+V]\psi=E\psi
\end{equation}
其中c为光速,$\bm{p}$为动量算符,$
\bm{\beta}=
\begin{bmatrix}
\bm{I} & 0 \\
0 & \bm{I}
\end{bmatrix}
$,\textbf{I}为单位阵,$
\bm{\alpha} = 
\begin{bmatrix}
\bm{\sigma} & 0 \\
0 & \bm{\sigma}
\end{bmatrix}
$,其中$\bm{\sigma}$为$2\times2$的Pauli自旋矢量
\begin{equation}
\bm{\sigma_x} = 
\begin{bmatrix}
0 & 1 \\
1 & 0
\end{bmatrix},
\bm{\sigma_y} =
\begin{bmatrix}
0 & -i \\
i & 0
\end{bmatrix},
\bm{\sigma_z} =
\begin{bmatrix}
1 & 0 \\
0 & -1
\end{bmatrix}
\end{equation}
他们都是酉矩阵。\par
显然,这个方程是一个旋量(spinor)方程,本征函数是自旋波函数。
\begin{equation}
\psi = 
\begin{bmatrix}
\psi_{L,\alpha} \\
\psi_{L,\beta} \\
\psi_{S,\alpha} \\
\psi_{S,\beta}
\end{bmatrix}
=
\begin{bmatrix}
\psi_{L} \\
\psi_{S}
\end{bmatrix}
\end{equation}
其中$\psi_L$是大分量,代表电子;$\psi_S$是小分量,代表正电子。\\
产生反电子的缘由正是因为相对论极限,当光速接近无穷时,$\psi_L = \psi_{Schrodinger}$,$\psi_S$不存在。\par
相对论量子化学和传统量化在理论表达上差别很大,可以参考doc目录里的《handbook of relativistic quantum chemistry compress》,出自刘文剑老师主编,其中尤其是x2c方法的描述非常详细。

\subsection{旋轨耦合哈密顿量}
  相对论哈密顿量可以写成
\begin{equation}
H_{rel}=H_{sf}+H_{sd}(\bm{\sigma})
\end{equation}
前者与Pauli矩阵无关,是标量部分;后者有关,其中最重要的部分是旋轨耦合(SOC):
$$
H_{sd}=\sum_{pq}[h_{sd}]_{pq}a^{\dagger}_pa_q, [h_{sd}]_{pq}=[h_{SO,1e}]_{pq}+[f_{SO,2e}]_{pq}
$$
可见旋轨耦合矩阵包括单电子积分和双电子积分两部分。双电子部分处理较困难,一般使用平均场(SOMF)处理,也可以使用有效原子电荷(Zeff)。
旋轨耦合矩阵元为$(I+J)*(I+J)$的矩阵,包括实部和虚部,如果是三-五重态,则为8*8。
$$
[\bm{H}_{SO}]_{IJ}=\langle\Psi_I|H_{sd}|\Psi_J\rangle
$$
$\bm{H}_{SO}$的定义不唯一,取决于用什么方法把Dirac方程变换为二分量。比较常用的方法有Breit-Pauli(最常用)、DKn、ZORA。
\newpage


\section{系间窜越速率}
\subsection{Marcus方程}
Marcus方程的形式为:
$$
\Delta G=\frac{\lambda}{4}\left(1+\frac{\Delta G_0}{\lambda}\right)^2
$$
其中$\Delta G$是反应活化能,$\Delta G_0$是Gibbs自由能,$\lambda$是重组能。\\
其中势能面近似为二次函数,使用四点法近似。其能量变化的表达式为:
$$
E=\frac{N_{A}{({\Delta}E_{ST}+\lambda)}^2}{4{\lambda}}
$$
其中${\Delta}E_{ST}$为三-五重态基态能量变化量。
阿伦尼乌斯公式:
$$
k=Ae^{-\frac{E}{RT}}
$$
其中指前因子A是一个自旋-轨道二者的耦合关系有关的量,其表达式可以写成:
$$
A=\frac{2\pi}{\hbar}{|SOCME}|^2\sqrt{\frac{1}{4\pi k_BT\lambda}}
$$
\subsection{误差分析}
因为态-态转换是一个非绝热过程,我们传统计算方法得不到交叉点电子结构,就得不到这个点的能量,也就没法求能垒,所以使用四点法估算,也有其他方法,比如thular的两态自旋混合(TSSM)模型。显然因为我们得不到非绝热势能面交叉点电子结构,也就无法精确态-态转化的耦合量。这个公式对于气相反应从原理上来说很精确,但我们的反应是溶液体系,再能量上有一个与溶液环境有关的分量lambda。如果结果是用作比较,而非绝对值,可以忽略,更精确也只能使用经验参数。此外,误差来源还有四点法估计能垒本身的问题;分子结构对旋轨耦合矩阵元的影响。
\newpage
\section{SOCME计算流程}
\subsection{输入和输出}

我们以五重态氧化亚铁为例,使用orca软件在casscf(6,6)下计算三-五重态之间的旋轨耦合矩阵元。
关键词为
\begin{verbatim}
%pal nprocs 12 end
%maxcore 3000
! TightSCF miniprint nopop
%casscf
 nel 6
 norb 6
   mult 5,3
  rel 
  dosoc true
  PrintLevel 3
  end
 maxiter 150
 CI
  MaxIter 200
 end
end
%method
 FrozenCore FC_NONE
end
%scf
 Thresh 1e-12
 Tcut 1e-14
end
\end{verbatim}
其中开启SOC计算的关键词为$dosoc true$;我把$mult$设置为$3,5$表示让程序搜索三重态基态和五重态基态,如果不设置的话,程序会自动判断到5,计算SOC时候自动识别到3;最终在.out文件中:
\begin{verbatim}
------------------------------------
NONZERO SOC MATRIX ELEMENTS (cm**-1)
------------------------------------

           Bra                       Ket       
<Block Root  S    Ms  | HSOC |  Block Root  S    Ms>    =  Real-part     Imaginary part
--------------------------------------------------------------------------------------
   1    0  1.0  1.0              0     0  2.0  2.0     -0.221       0.287
   1    0  1.0  1.0              0     0  2.0  0.0     -0.090      -0.117
   1    0  1.0  0.0              0     0  2.0  1.0     -0.156   	0.203
   1    0  1.0  0.0              0     0  2.0  0.0     -0.000   	0.000
   1    0  1.0  0.0              0     0  2.0 -1.0     -0.156      -0.203
   1    0  1.0 -1.0              0     0  2.0  0.0     -0.090   	0.117
   1    0  1.0 -1.0              0     0  2.0 -2.0     -0.221      -0.287

\end{verbatim}
可以看到,程序输出了非零矩阵元,分别输出了实部和虚部,我们公式要用的SOCME是其模平方:
$$
\sqrt{(-0.221)^2+(0.287)^2+( -0.090)^2+( -0.117)^2+(-0.156)^2+(0.203)^2+(-0.156)^2+(      -0.203)^2+( -0.090)^2+(0.117)^2+(-0.221)^2+( -0.287)^2}
$$
\subsection{其他}
\begin{itemize}
  \item ORCA的CASSCF收敛性并不好,对于大体系建议使用Pyscf/OpenMolcas/Molpro得到收敛的CASSCF自然轨道,传给ORCA
  \item 建议使用mokit及其小程序选轨道和传轨道(\href{https://gitlab.com/jxzou/mokit}{mokit}) 
\end{itemize}
\newpage
\section{ISC的计算}
使用我用matlab写的计算器可以很方便的实现计算,程序附带抛物线近似和四点法示意图。

\newpage
\section{后记}
doc中ppt是我组会的ppt,exe是我打包好的matlab计算期安装包,不要求电脑中有matlab,安装即用,支持三种能量单位,也可以当个单位转换器使用。input是我例子中的FeO(V)的SOCME计算输入输出文件(输入文件是fch2mkl制作的),可以用其中的gbw文件作为初猜重复文中的数据。
\end{document}
