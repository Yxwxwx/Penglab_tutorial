\documentclass[12pt, a4paper, oneside]{ctexart}
\usepackage{amsmath, amsthm, amssymb, appendix, bm, graphicx, hyperref, mathrsfs, xcolor}
\usepackage{listings}
\usepackage{ctex}
% 用来设置附录中代码的样式

\lstset{
    basicstyle          =   \sffamily,          % 基本代码风格
    keywordstyle        =   \bfseries,          % 关键字风格
    commentstyle        =   \rmfamily\itshape,  % 注释的风格,斜体
    stringstyle         =   \ttfamily,  % 字符串风格
    flexiblecolumns,                % 别问为什么,加上这个
    numbers             =   left,   % 行号的位置在左边
    showspaces          =   false,  % 是否显示空格,显示了有点乱,所以不现实了
    numberstyle         =   \zihao{-5}\ttfamily,    % 行号的样式,小五号,tt等宽字体
    showstringspaces    =   false,
    captionpos          =   t,      % 这段代码的名字所呈现的位置,t指的是top上面
    frame               =   lrtb,   % 显示边框
}

\lstdefinestyle{Python}{
    language        =   Python, % 语言选Python
    basicstyle      =   \zihao{-5}\ttfamily,
    numberstyle     =   \zihao{-5}\ttfamily,
    keywordstyle    =   \color{blue},
    keywordstyle    =   [2] \color{teal},
    stringstyle     =   \color{magenta},
    commentstyle    =   \color{red}\ttfamily,
    breaklines      =   true,   % 自动换行,建议不要写太长的行
    columns         =   fixed,  % 如果不加这一句,字间距就不固定,很丑,必须加
    basewidth       =   0.5em,
}

\title{\textbf{一个简单的Hartree-Fock代码基于Pyscf}}
\author{Sun Xinyu\\sunxinyu347@gmail.com}
\date{\today}
\linespread{1}


\begin{document}

\maketitle

\setcounter{page}{0}
\maketitle
\thispagestyle{empty}

\newpage
\pagenumbering{Roman}
\setcounter{page}{1}
\tableofcontents
\newpage
\setcounter{page}{1}
\pagenumbering{arabic}

\section{前言}

  Hartree-Fock(后文简称HF)方法是量子化学最经典的波函数方法,如今常用来为后续高级别算法提供初猜、选择活性空间等。我们仍有必要学习HF代码,其中的思想和专有名词是应用量子化学计算的必要知识储备,让初学者对量化软件的逻辑有一个基础且必须的了解,避免糊算乱算,并且在自己的计算过程中遇到的错误能有合理的解释和解决办法。\par
这是南开大学彭谦课题组新人入组手册系列之一,gitlab地址为$https://github.com/Yxwxwx/Penglab\_tutorial$
\newpage

\section{Hartree-Fock理论}

  在这里我假设大家已经系统上过“线性代数”这门课,阅读过了Szabo的《Modern Quantum Chemistry》,至少也是阅读了Levine的《Quantum Chemistry》,能看懂我后续的公式。如果看不懂,就需要再仔细看书了。

\subsection{高斯型基函数(GTOs)}
  我们使用的基组函数一般都是GTO,其数学表达形式为$$|\phi^{GTO}\rangle=\left(\frac{2a}{\pi}\right)^{3/4}e^{-a{\bm{r}}^2}$$我们成为原始的Gaussian函数(primitive);比如STO-3G对于H原子,使用三个GTO拟合一个STO,那么Gaussian函数的线性组合(CGTO)可以写为
\begin{equation}
\begin{aligned}
    |\phi^{CGTO}(r)\rangle=&d_1\times\phi^{GTO}(a_1,\bm{r}) \\
    &+d_2\times\phi^{GTO}(a_2,\bm{r}) \\
    &+d_3\times\phi^{GTO}(a_3,\bm{r})\nonumber
\end{aligned}
\end{equation}
其中的系数$d$和指数$a$通过读取基组文件得到。
\begin{verbatim}
BASIS "ao basis" PRINT
#BASIS SET: (3s) -> [1s]
H    S
      3.42525091             0.15432897       
      0.62391373             0.53532814       
      0.16885540             0.44463454     
\end{verbatim}

\subsection{电子积分}
  这里只介绍HF方法中使用到的电子积分,并不展开其计算公式。
\begin{itemize}
  \item 单电子积分
  \begin{itemize}
    \item 重叠积分S:$S_{pq}=\langle\psi_p|\psi_q\rangle$
    \item 动能积分T:$T_{pq}=\langle\psi_p|-\frac{1}{2}\nabla^2|\psi_q\rangle$
    \item 核-电子势能积分V:$V_{pq}=\langle\psi_p|\frac{1}{r_C}|\psi_q\rangle$
	\item 核-哈密顿矩阵H:H=
  \end{itemize}
  \item 双电子积分I:$I_{pqrs} = \int \int d\mathbf{r_1} d\mathbf{r_2} \phi_p^*(\mathbf{r_1}) \phi_q(\mathbf{r_1}) \phi_r^*(\mathbf{r_2}) \phi_s(\mathbf{r_2})$
\end{itemize}
对于水分子在STO-3G下,共有10个电子、7个分子轨道,所以单电子积分为7*7的对称阵,双电子积分为7*7*7*7的四维矩阵,显然,双电子积分是自洽场计算中最耗时的部分,此处留个疑问:\textbf{存储四维矩阵显然浪费,且处理四维矩阵更耗时,计算程序采用那些策略优化呢?}
\subsection{Roothan方程}
  HF方法,又称为自洽场方法,目标为解一个伪特征值矩阵方程(pseudo-eigenvalue matrix equation):
$$\mathbf{FC}=\mathbf{SC}\epsilon$$
通过迭代的方法求解稀疏矩阵\textbf{C}得到能量本征值$\epsilon_{i}$,其中\textbf{S}为重叠积分,\textbf{F}为Fock矩阵,其通过电子积分得到:
$$F_{pq}=\mathbf{H}+2(pq|rs)D_{rs}-(pr|qs)D_{rs}$$
其中$2(pq|rs)D_{rs}$被称为库伦积分(\textbf{J}),$-(pq|rs)D_{rs}$被称为交换积分(\textbf{K}),矩阵\textbf{D}为密度矩阵(density matrix):
$$D_{pq}=\sum_{i}C_{pi}C_{qi}$$
Hartree-Fock能量为:
$$E_{elec}=\left(F_{pq}+H_{pq}\right)D_{pq}$$

\newpage

\section{程序编写}
\subsection{环境准备}
  对于开发来说,尤其是计算化学程序,Linux绝对是最好的选择,相比Windows下需要考虑的更少,开发效率更高、更灵活。显然大多数电脑系统还是Windows,所以建议大家安装WSL2(\href{https://mp.weixin.qq.com/s/DoK08_Qpvt28ZjAZQa8MZA}{安装教程}),安装好gcc和gfortran,建议使用Anaconda简化python库文件的管理,我们后续代码需要Numpy。其次建议使用MacOS系统。本文所有代码均在WSL2中运行,Python版本为3.10\par
  教程使用Python,当然使用什么语言无所谓,C/C++、Fortran都可以,他们的效率可能更高,但需要一定熟练度,否则不一定快于Numpy(因为其底层还是用C/Fortran写的),使用Python我认为对新手更Friendly。我会提供单、双电子积分,可以直接load,如果想跑其他结构,建议安装Pyscf。\par
对于Python语法,我不会用很复杂,因为只涉及矩阵处理,但也没精力再写一遍Numpy教程。这里假设读者水平为大学上过编程课程,会使用Chat-GPT等AI大模型问问题,基本上课余时间一周肯定能写出来!
\subsection{程序流程}
根据Szabo和Ostlund书中146页描述,自洽场迭代步骤为:
\begin{itemize}
  \item 步骤
  \begin{enumerate}
	\item 得到一个密度矩阵初猜
	\item 根据电子积分和初猜构建Fock矩阵
	\item 对角化Fock矩阵
	\item 选择占据轨道并计算新的密度矩阵
	\item 计算HF能量
	\item 计算误差判断是否收敛
	\begin{itemize}
	  \item 如果不收敛,使用新的密度矩阵继续
	  \item 如果收敛,输出能量
	\end{itemize}
  \end{enumerate}
\end{itemize}
\subsection{代码实现}
我们的任务目标是:根据提供的单、双电子积分,计算$H_2O$分子在$STO-3G$基组下的HF电子能量,参考值:$-84.1513215474753$\\
载入环境
\begin{lstlisting}[style = Python]
import numpy as np
import scipy 
\end{lstlisting}
将.npy二进制文件和python脚本文件放在同一文件夹中,载入单电子积分和双电子积分,同时获取轨道数nao。我们使用一个单位阵作为初猜,对应步骤1.值得注意,这是一种很粗糙的制作初猜的方式,不同的计算化学程序又不同的制作初猜的方法,初猜越准确,自洽场收敛越快。
\begin{lstlisting}[style = Python]
overlap_matrix = np.load("overlap.npy")
H = np.load("core_hamiltonian.npy")
int2e = np.load("int2e.npy")
nao = len(overlap_matrix[0])
assert nao == len(int2e[0])
dm = np.eye(nao)
\end{lstlisting}
下面我们需要些两个函数,用于构建库伦积分\textbf{J}和交换积分\textbf{K},传入函数的是密度矩阵$dm$,对应步骤2.因为双电子积分是四维矩阵,所以应该是四重循环。根据上文的公式,遍历壳层$p, q, r, s$
\begin{lstlisting}[style = Python]
# Calculate the coulomb matrices from density matrix
def get_j(dm):
    J = np.zeros((nao, nao))  # Initialize the Coulomb matrix

    # Loop over all indices of the Coulomb matrix
    for p in range(nao):
        for q in range(nao):
            # Calculate the Coulomb integral for indices (p,q)
            for r in range(nao):
                for s in range(nao):
                    J[p, q] += dm[r, s] * int2e[p, q, r, s]

    return J

# Calculate the exchange  matrices from density matrix
def get_k(dm):
    K = np.zeros((nao, nao))  # Initialize the K matrix

    # Loop over all indices of the K matrix
    for p in range(nao):
        for q in range(nao):
            # Calculate the K integral for indices (p,q)
            for r in range(nao):
                for s in range(nao):
                    K[p, q] += dm[r, s] * int2e[p, r, q, s]

    return K
\end{lstlisting}
我们还需要一个函数用于构建新的密度矩阵$dm$,将Fock矩阵传入函数。根据密度矩阵的公式,我们要遍历所有占据轨道的系数矩阵。对于本体系共10个电子,也就是5个占据轨道和2个空轨道。在这个函数中,我们要先完成步骤3,将Fock矩阵对角化,得到系数矩阵\textbf{C}。我们需要将所有原子轨道基函数正交化,一个可以的方法是对称正交化。例如定义一个矩阵\textbf{A},满足:
$$\mathbf{A}^{\dagger}\mathbf{S}\mathbf{A}=1$$
令$\mathbf{A}=\mathbf{S}^{-1/2}$,于是有:
$$\mathbf{A}^{\dagger}\mathbf{S}\mathbf{A}=\mathbf{S}^{1/2}\mathbf{S}\mathbf{S}^{-1/2}=\mathbf{S}^{-1/2}\mathbf{S}^{1/2}=\mathbf{S}^{0}=1$$
我们借助矩阵\textbf{A}将Roothan方程转换为本征方程(canonical eigenvalue equation),令$\mathbf{C}=\mathbf{A}\mathbf{C}^{\prime}$
$$\mathbf{F}\mathbf{A}\mathbf{C}^{\prime}=\mathbf{S}\mathbf{A}\mathbf{C}^{\prime}\epsilon$$
$$\mathbf{A}^{\dagger}(\mathbf{F}\mathbf{A}\mathbf{C}^{\prime})=\mathbf{A}^{\dagger}(\mathbf{S}\mathbf{A}\mathbf{C}^{\prime})\epsilon$$
$$(\mathbf{A}^{\dagger}\mathbf{F}\mathbf{A})\mathbf{C}^{\prime}=(\mathbf{A}^{\dagger}(\mathbf{S}\mathbf{A})\mathbf{C}^{\prime}\epsilon$$
$$\mathbf{F}^{\prime}\mathbf{C}^{\prime}=\mathbf{C}^{\prime}\epsilon$$
于是我们可以通过将$\mathbf{F}^{\prime}$矩阵对角化得到$\mathbf{C}^{\prime}$,再将其转换回$\mathbf{C}$通过$\mathbf{C}=\mathbf{A}\mathbf{C}^{\prime}$
可以通过Scipy库实现,我们使用了scipy.linalg.fractional\_matrix\_power()函数和np.linalg.eigh()函数
\begin{lstlisting}[style = Python]
# Calculate the density matrix
def get_dm(fock, nocc):
    dm = np.zeros((nao, nao))
    S = overlap_matrix
    A = scipy.linalg.fractional_matrix_power(S, -0.5)
    F_p = A.T @ fock @ A
    eigs, coeffsm = np.linalg.eigh(F_p)

    c_occ = A @ coeffsm
    c_occ = c_occ[:, :nocc]
    for i in range(nocc):
        for p in range(nao):
            for q in range(nao):
                dm[p, q] += c_occ[p, i] * c_occ[q, i]
    return dm
\end{lstlisting}
OK!准备工作已经充足,可以开始自洽场迭代了!对应步骤5、6\\
我们首先要确定收敛限和最大迭代次数。收敛限即最后收敛能量与上一次循环的HF能量变化小于一个值,我们设置为$1.0e-10$,最大迭代次数为40次同时将能量初始化
\begin{lstlisting}[style = Python]
# Maximum SCF iterations
max_iter = 100
E_conv = 1.0e-10
# SCF & Previous Energy
SCF_E = 0.0
E_old = 0.0
\end{lstlisting}
根据步骤6书写循环
\begin{lstlisting}[style = Python]
for scf_iter in range(1, max_iter + 1):
    # GET Fock martix
    F = H + 2 * get_j(dm) - get_k(dm)
    assert F.shape == (nao, nao)

    SCF_E = np.sum(np.multiply((H + F), dm))
    dE = SCF_E - E_old
    print('SCF Iteration %3d: Energy = %4.16f dE = % 1.5E' % (scf_iter, SCF_E, dE))

    if (abs(dE) < E_conv):
        print("SCF convergence! Congrats")
        break
    E_old = SCF_E

    dm = get_dm(F, 5)

assert(np.abs(SCF_E + 84.1513215474753) < 1.0e-10)
\end{lstlisting}
\newpage
\section{后记}
完整的代码我上传在\href{https://github.com/Yxwxwx/Penglab_tutorial}{gitlab}上,从上面可以下载二进制积分文件,npy和源代码。本问的代码非常简单,只有几十行,而且有非常多可以改良的地方,比如\textbf{get\_j}函数中使用了四重循环,这在python中是灾难性的代码;而且也并未考虑积分对称性。我列出几个可以继续思考的地方:
\begin{enumerate}
	\item 结合参考书,实现UHF代码
	\item 使用np.einsum代替循环和一些内置函数以提高效率
	\item 考虑双电子积分的八重对称性加速构建Fock矩阵
	\item 使用DIIS技术加速SCF收敛
	\item 使用Cython,C/C++,Fortran等静态编程语言重写代码以加速
	\item 安装并学习Pyscf,尝试计算更多电子结构
	\item ...
\end{enumerate}\par
能自己写一个计算化学代码并且和自己常用的计算化学软件对应上,其实是一个很有成就感的过程。Keep coding!Keep thinking!



\newpage
\begin{thebibliography}{99}
    \bibitem{a}Szabo and Ostlund. \emph{Modern Quantum Chemistry: Introduction to Advanced Electronic Structure Theory}[M]. New York:Dover Publications,1996.
\end{thebibliography}

\newpage

\begin{appendices}
    \renewcommand{\thesection}{\Alph{section}}
    \section{完整代码}
\begin{lstlisting}[style = Python]
import numpy as np
import scipy

overlap_matrix = np.load("overlap.npy")
H = np.load("core_hamiltonian.npy")
int2e = np.load("int2e.npy")
nao = len(overlap_matrix[0])
assert nao == len(int2e[0])
assert int2e.shape == (nao, nao, nao, nao)
dm = np.eye(nao)

# Calculate the coulomb matrices from density matrix
def get_j(dm):
    J = np.zeros((nao, nao))  # Initialize the Coulomb matrix

    # Loop over all indices of the Coulomb matrix
    for p in range(nao):
        for q in range(nao):
            # Calculate the Coulomb integral for indices (p,q)
            for r in range(nao):
                for s in range(nao):
                    J[p, q] += dm[r, s] * int2e[p, q, r, s]

    return J

# Calculate the exchange  matrices from density matrix
def get_k(dm):
    K = np.zeros((nao, nao))  # Initialize the K matrix

    # Loop over all indices of the K matrix
    for p in range(nao):
        for q in range(nao):
            # Calculate the K integral for indices (p,q)
            for r in range(nao):
                for s in range(nao):
                    K[p, q] += dm[r, s] * int2e[p, r, q, s]

    return K

# Calculate the density matrix
def get_dm(fock, nocc):
    dm = np.zeros((nao, nao))
    S = overlap_matrix
    A = scipy.linalg.fractional_matrix_power(S, -0.5)
    F_p = A.T @ fock @ A
    eigs, coeffsm = np.linalg.eigh(F_p)

    c_occ = A @ coeffsm
    c_occ = c_occ[:, :nocc]
    for i in range(nocc):
        for p in range(nao):
            for q in range(nao):
                dm[p, q] += c_occ[p, i] * c_occ[q, i]
    return dm
# Maximum SCF iterations
max_iter = 100
E_conv = 1.0e-10
# SCF & Previous Energy
SCF_E = 0.0
E_old = 0.0
for scf_iter in range(1, max_iter + 1):
    # GET Fock martix
    F = H + 2 * get_j(dm) - get_k(dm)
    assert F.shape == (nao, nao)

    SCF_E = np.sum(np.multiply((H + F), dm))
    dE = SCF_E - E_old
    print('SCF Iteration %3d: Energy = %4.16f dE = % 1.5E' % (scf_iter, SCF_E, dE))

    if (abs(dE) < E_conv):
        print("SCF convergence! Congrats")
        break
    E_old = SCF_E

    dm = get_dm(F, 5)

assert(np.abs(SCF_E + 84.1513215474753) < 1.0e-10)
\end{lstlisting}
\end{appendices}

\end{document}
