\documentclass[12pt, a4paper, oneside]{ctexart}
\usepackage{amsmath, amsthm, amssymb, appendix, bm, graphicx, hyperref, mathrsfs, xcolor}
\usepackage{listings}
\usepackage{ctex}
% 用来设置附录中代码的样式

\lstset{
    basicstyle          =   \sffamily,          % 基本代码风格
    keywordstyle        =   \bfseries,          % 关键字风格
    commentstyle        =   \rmfamily\itshape,  % 注释的风格,斜体
    stringstyle         =   \ttfamily,  % 字符串风格
    flexiblecolumns,                % 别问为什么,加上这个
    numbers             =   left,   % 行号的位置在左边
    showspaces          =   false,  % 是否显示空格,显示了有点乱,所以不现实了
    numberstyle         =   \zihao{-5}\ttfamily,    % 行号的样式,小五号,tt等宽字体
    showstringspaces    =   false,
    captionpos          =   t,      % 这段代码的名字所呈现的位置,t指的是top上面
    frame               =   lrtb,   % 显示边框
}

\lstdefinestyle{Python}{
    language        =   Python, % 语言选Python
    basicstyle      =   \zihao{-5}\ttfamily,
    numberstyle     =   \zihao{-5}\ttfamily,
    keywordstyle    =   \color{blue},
    keywordstyle    =   [2] \color{teal},
    stringstyle     =   \color{magenta},
    commentstyle    =   \color{red}\ttfamily,
    breaklines      =   true,   % 自动换行,建议不要写太长的行
    columns         =   fixed,  % 如果不加这一句,字间距就不固定,很丑,必须加
    basewidth       =   0.5em,
}

\title{\textbf{Hartree-Fock理论}}
\author{Sun Xinyu\\sunxinyu347@gmail.com}
\date{\today}
\linespread{1}


\begin{document}

\maketitle

\setcounter{page}{0}
\maketitle
\thispagestyle{empty}

\newpage
\pagenumbering{Roman}
\setcounter{page}{1}
\tableofcontents
\newpage
\setcounter{page}{1}
\pagenumbering{arabic}

\section{前言}

  Hartree-Fock(后文简称HF)方法是量子化学最经典的波函数方法,如今常用来为后续高级别算法提供初猜、选择活性空间等。我们仍有必要学习HF代码,其中的思想和专有名词是应用量子化学计算的必要知识储备,让初学者对量化软件的逻辑有一个基础且必须的了解,避免糊算乱算,并且在自己的计算过程中遇到的错误能有合理的解释和解决办法。\par
这是南开大学彭谦课题组新人入组手册系列之一,gitlab地址为$https://github.com/Yxwxwx/Penglab\_tutorial$
\newpage

\section{Hartree-Fock理论}
本章要求熟练掌握线性代数相关知识,熟悉《结构化学》中,量子力学基础部分。

\subsection{高斯型基函数(GTOs)}
  我们使用的基组函数一般都是GTO,其数学表达形式为$$|\phi^{GTO}\rangle=\left(\frac{2a}{\pi}\right)^{3/4}e^{-a{\bm{r}}^2}$$我们成为原始的Gaussian函数(primitive);比如STO-3G对于H原子,使用三个GTO拟合一个STO,那么Gaussian函数的线性组合(CGTO)可以写为
\begin{equation}
\begin{aligned}
    |\phi^{CGTO}(r)\rangle=&d_1\times\phi^{GTO}(a_1,\bm{r}) \\
    &+d_2\times\phi^{GTO}(a_2,\bm{r}) \\
    &+d_3\times\phi^{GTO}(a_3,\bm{r})\nonumber
\end{aligned}
\end{equation}
其中的系数$d$和指数$a$通过读取基组文件得到。
\begin{verbatim}
BASIS "ao basis" PRINT
#BASIS SET: (3s) -> [1s]
H    S
      3.42525091             0.15432897       
      0.62391373             0.53532814       
      0.16885540             0.44463454     
\end{verbatim}

\subsection{多电子Hartree-Fock近似}
\subsubsection{Fock算符}
根据变分原理,对于Slater行列式形式的波函数中,最优的波函数对应的是最低的能量:
\begin{equation}
E_0=\langle\Psi_0|\mathbf{H}|\Psi_0\rangle
\end{equation}
在单电子近似下,我们假设一个有效的单电子算符$f(i)$,被称为Fock算符,其形式为:
\begin{equation}
f(i)=-\frac12\Delta_i^2-\sum_{A-1}^M\frac{Z_A}{r_{iA}}+v^{HF}(i)
\end{equation}
$v^{HF}(i)$是$i$电子和其他电子的相互作用产生的平均势。所谓“自洽场”中的场,可以简单理解为这个平均势。Hartree-Fock近似的精髓在于将复杂的多电子问题转化为单电子问题,单电子问题使用平均的方式处理复杂的电子排斥。\\
Fock算符是轨道的本征函数:
\begin{equation}
f|\chi_a\rangle=\epsilon_a|\chi_a\rangle
\end{equation}
$\chi_a$是自旋轨道波函数,后文会经常用到。
\subsubsection{自旋轨道下的Hartree-Fock能量方程}
基态波函数的Slater行列式形式为:
\begin{equation}
\Psi_0=\frac1{2^{1/2}}
\begin{vmatrix}
\chi_1(x_1) & \chi_2(x_1)\\
\chi_1(x_2) & \chi_2(x_2)
\end{vmatrix}
=|\chi_1\chi_2\rangle
\end{equation}

以极小基$H_2$模型考虑。在Born-Oppenheimer近似下,双电子体系的哈密顿量为:
\begin{equation}
\begin{aligned}
	\mathbf{H}&= \left(-\frac12\Delta_1^2-\sum_A\frac{Z_A}{r_{1A}}\right)+\left(-\frac12\Delta_2^2-\sum_A\frac{Z_A}{r_{2A}}\right)+\frac{1}{r_{12}}\\
			  &=h(1)+h(2)+\frac{1}{r_{12}}
\end{aligned}
\end{equation}
其中$h(1)$就是电子1的“核哈密顿量”,表示电子在核的势场中的动能和势能,对应后文的“单电子积分”,$\frac{1}{r_{12}}$对应双电子部分。为方便,我们将总哈密顿分成单电子和双电子部分:
\begin{equation}
\begin{aligned}
	\mathbf{O}_1&=h(1)+h(2)\\
	\mathbf{O}_2&=r_{12}^{-1}
\end{aligned}
\end{equation}
矩阵元$\langle\Psi_0|\mathbf{H}|\Psi_0\rangle=\langle\Psi_0|\mathbf{O_1}|\Psi_0\rangle+\langle\Psi_0|\mathbf{O_2}|\Psi_0\rangle$
\begin{equation}
\begin{aligned}
\langle\Psi_0|h(1)|\Psi_0\rangle&=\int dx_1dx_2[2^{\frac{-1}{2}}(\chi_1(x_1)\chi_2(x_2)-\chi_2(x_1)\chi_1(x_2))]^*\\
								&\times h(r_1)[2^{\frac{-1}{2}}(\chi_1(x_1)\chi_2(x_2)-\chi_2(x_1)\chi_1(x_2))]
\end{aligned}
\end{equation}
\begin{equation}
\begin{aligned}
=\frac12\int dx_1dx_2&[\chi_1^*(x_1)\chi_2^*(x_2)h(r_1)\chi_1(x_1)\chi_2(x_2)-\chi_1^*(x_1)\chi_2^*(x_2)h(r_1)\chi_2(x_1)\chi_1(x_2)\\
&-\chi_2^*(x_1)\chi_1^*(x_2)h(r_1)\chi_1(x_1)\chi_2(x_2)+\chi_2^*(x_1)\chi_1^*(x_2)h(r_1)\chi_2(x_1)\chi_1(x_2)
\end{aligned}
\end{equation}
根据自旋轨道的正交归一性,将$x_2$积掉,公式(6)只有两项:
\begin{equation}
\begin{aligned}
\langle\Psi_0|h(1)|\Psi_0\rangle&=\frac12\int dx_1\chi_1^*(x_1)h(r_1)\chi_1(x_1)+\frac12\int dx_1\chi_2^*(x_1)h(r_1)\chi_2(x_1)
\end{aligned}
\end{equation}
对于重复操作,可得:$\langle\Psi_0|h(1)|\Psi_0\rangle=\langle\Psi_0|h(2)|\Psi_0\rangle$
则:
\begin{equation}
\begin{aligned}
\langle\Psi_0|\mathbf{O_1}|\Psi_0\rangle=\int dx_1\chi_1^*(x_1)h(r_1)\chi_1(x_1)+\int dx_1\chi_2^*(x_1)h(r_1)\chi_2(x_1)
\end{aligned}
\end{equation}
上式被称为\textbf{单电子积分},使用$\langle i|h|j\rangle=\langle\chi_i|h|\chi_j\rangle=\int dx_1\chi_1^*(x_1)h(r_1)\chi_1(x_1)$简化表示,则有:
\begin{equation}
\begin{aligned}
\langle\Psi_0|\mathbf{O_1}|\Psi_0\rangle=\langle1|h|1\rangle+\langle 2|h|2\rangle
\end{aligned}
\end{equation}
现在计算$O_2$矩阵元:
\begin{equation}
\begin{aligned}
\langle\Psi_0|\mathbf{O_2}|\Psi_0\rangle&=\int dx_1dx_2[2^{\frac{-1}{2}}(\chi_1(x_1)\chi_2(x_2)-\chi_2(x_1)\chi_1(x_2))]^*\\
								&\times r_{12}^{-1}[2^{\frac{-1}{2}}(\chi_1(x_1)\chi_2(x_2)-\chi_2(x_1)\chi_1(x_2))]
\end{aligned}
\end{equation}
\begin{equation}
\begin{aligned}
=\frac12\int dx_1dx_2&[\chi_1^*(x_1)\chi_2^*(x_2)r_{12}^{-1}\chi_1(x_1)\chi_2(x_2)-\chi_1^*(x_1)\chi_2^*(x_2)r_{12}^{-1}\chi_2(x_1)\chi_1(x_2)\\
&-\chi_2^*(x_1)\chi_1^*(x_2)r_{12}^{-1}\chi_1(x_1)\chi_2(x_2)+\chi_2^*(x_1)\chi_1^*(x_2)r_{12}^{-1}\chi_2(x_1)\chi_1(x_2)
\end{aligned}
\end{equation}
显然$r_{12}^{-1}=r_{21}^{-1}$,上式中的积分变量可以交换,即第一项和第四项相同,第二项与第三项相同,因此:
\begin{equation}
\begin{aligned}
\langle\Psi_0|\mathbf{O_2}|\Psi_0\rangle&=\int dx_1dx_2\chi_1^*(x_1)\chi_2^*(x_2)r_{12}^{-1}\chi_1(x_1)\chi_2(x_2)\\
&-\int dx_1dx_2\chi_1^*(x_1)\chi_2^*(x_2)r_{12}^{-1}\chi_2(x_1)\chi_1(x_2)
\end{aligned}
\end{equation}
与单电子积分类似,我们同样以$$\langle ij|kl\rangle=\langle\chi_i\chi_j|\chi_k\chi_l\rangle=\int dx_1dx_2\chi_i^*(x_1)\chi_j^*(x_2)r_{12}^{-1}\chi_k(x_1)\chi_l(x_2)$$简化表示。那么就有:
\begin{equation}
\langle\Psi_0|\mathbf{O_2}|\Psi_0\rangle=\langle 12|12\rangle-\langle 12|21\rangle
\end{equation}
所以,Hartree-Fock基态能量为:
\begin{equation}
\begin{aligned}
E_0=\langle\Psi_0|\mathbf{H}|\Psi_0\rangle&=\langle\Psi_0|\mathbf{O_1}+\mathbf{O_2}|\Psi_0\rangle\\
  										  &=\langle 1|h|1\rangle+\langle 2|h|2\rangle+\langle 12|12\rangle-\langle 12|21\rangle
\end{aligned}
\end{equation}
有时也常用\textbf{反对称双电子积分}$\langle ij||kl\rangle ij|kl\rangle-\langle ij|lk\rangle$
上述使用的积分符号也被称为\textbf{物理学家符号},为方便,本文所有公式推导均是在物理学家符号下。
双电子积分具有以下对称性:
\begin{equation}
\begin{aligned}
\langle ij|kl\rangle&=\langle ji|lk\rangle\\
\langle ij|kl \rangle&=\langle kl|ij\rangle^*
\end{aligned}
\end{equation}
对于孤立体系、不考虑相对论效应的情况下,双电子积分的矩阵元均为实数,所以双电子积分具有\textbf{八重对称性}。\\
还有一种符号计法,被称为\textbf{化学家符号},自旋轨道中使用$[]$表示,和物理学家符号的关系是:
\begin{equation}
\begin{aligned}
\langle ij|kl\rangle = \langle \chi_i\chi_j|\chi_k\chi_l\rangle = [ik|jl]
\end{aligned}
\end{equation}
对于空间轨道,化学家符号使用$()$表示。事实上,常用的量子化学电子积分库Libint2,Libcint都是在化学家符号下计算的。

\subsubsection{空间轨道下的Hartree-Fock能量方程}
使用自旋轨道可以简化公式推导,但实际计算中自旋函数$\alpha,\beta$必须呗积分掉才能将其约化为可数值计算的空间轨道和积分。(实际上,对于闭壳层,自旋轨道比自旋轨道多浪费一个维度的资源)。\par
自旋轨道下Hartree-Fock基态能量为:
\begin{equation}
\begin{aligned}
E_0=\langle \chi_1|h|\chi_1\rangle+\langle \chi_2|h|\chi_2\rangle+\langle \chi_1\chi_2|\chi_1\chi_2\rangle-\langle \chi_1\chi_2|\chi_2\chi_1\rangle
\end{aligned}
\end{equation}
我们知道,空间轨道和自旋轨道的关系为:
\begin{equation}
\begin{aligned}
\chi_1(x)&=\psi_1(r)\alpha(w)=\psi_1 \\
\chi_2(x)&=\psi_1(r)\beta(w)=\overline{\psi_1}
\end{aligned}
\end{equation}
联立上述方程,可以得到空间轨道下的Hartree-Fock能量:
\begin{equation}
\begin{aligned}
E_0=\langle \psi_1|h|\psi_1\rangle+\langle \overline{\psi_1}|h|\overline{\psi_1}\rangle+\langle \psi_1\overline{\psi_1}|\psi_1\overline{\psi_1}\rangle-\langle \psi_1\overline{\psi_1}|\overline{\psi_1}\psi_1\rangle
\end{aligned}
\end{equation}
对于单电子积分:
\begin{equation}
\begin{aligned}
\langle \overline{\psi_1}|h|\overline{\psi_1}\rangle=\int dr\psi_1^*(r_1)\beta^*h(r_1)\psi_1(r_1)\beta
\end{aligned}
\end{equation}
对于非相对论下,单电子算符是不依赖自旋的,根据正交归一性,$\langle \beta|\beta\rangle=1$,所以显然:
\begin{equation}
\begin{aligned}
\langle \overline{\psi_1}|h|\overline{\psi_1}\rangle=\langle \psi_1|h|\psi_1\rangle
\end{aligned}
\end{equation}
对于双电子积分的第一项:
\begin{equation}
\begin{aligned}
\langle \psi_1\overline{\psi_1}|\psi_1\overline{\psi_1}\rangle=&\int dr_1dr_2dw_1dw_2\psi_1^*(r_1)\alpha^*(w_1)\psi_1(r_2)\alpha(w_2) r_{12}^{-1}\\
&\times\psi_1^*(r_1)\beta^*(w_1)\psi_1(r_2)\beta(w_2)
\end{aligned}
\end{equation}
同样把$\alpha,\beta$积掉,得到:
\begin{equation}
\begin{aligned}
\langle \psi_1\overline{\psi_1}|\psi_1\overline{\psi_1}\rangle&=\int dr_1dr_2\psi_1^*(r_1)\psi_1(r_2) r_{12}^{-1}\psi_1^*(r_1)\psi_1(r_2)\\
&=\langle \psi_1\psi_1|\psi_1\psi_1\rangle
\end{aligned}
\end{equation}
对于双电子积分的第二项:
\begin{equation}
\begin{aligned}
\langle \psi_1\overline{\psi_1}|\overline{\psi_1}\psi_1\rangle=&\int dr_1dr_2\psi_1^*(r_1)\alpha^*(w_1)\psi_1^*(r_2)\beta^*(w_2) r_{12}^{-1}\\
&\times\psi_1(r_1)\beta(w_1)\psi_1(r_2)\alpha(w_2)=0
\end{aligned}
\end{equation}
这是因为正交性$\langle \alpha|\beta\rangle=\langle \beta|\alpha\rangle=0$\\
所以,空间轨道下的极小基H2的Hartree-Fock基态能量为:
\begin{equation}
\begin{aligned}
E_0&=2\langle\psi_1|h|\psi_1\rangle+\langle\psi_1\psi_1|\psi_1\psi_1\rangle\\
   &=2\langle1|h|1\rangle+\langle 11|11\rangle
\end{aligned}
\end{equation}
通常下,空间轨道的公式会比自旋轨道下的更紧凑,更好编程。
\subsubsection{N电子体系闭壳层限制性Hartree-Fock能量方程}
类似H2极小基的莫函数,N电子系统的Hartree-Fock波函数为:
\begin{equation}
\begin{aligned}
|\Psi_0\rangle&=|\chi_1\chi_2\chi_3\chi_4\dots\chi_{N-1}\chi_N\rangle\\
			&=|\psi_1\overline{\psi_1}\psi_2\overline{\psi_2}\dots\psi_{N/2}\overline{\psi_{N/2}}\rangle
\end{aligned}
\end{equation}
对于自旋轨道,Hartree-Fock能量方程为:
\begin{equation}
\begin{aligned}
E_0=\sum_i^N\langle i|h|i\rangle+\frac12\sum_i^N\sum_j^N\langle ij|ij\rangle-\langle ij|ji\rangle
\end{aligned}
\end{equation}
其中的双电子部分,共N/2对电子,故有$\frac12$\par
自旋轨道波函数包括N/2个$\alpha$自旋轨道和N/2个$\beta$自旋轨道,我们将自旋轨道的求和分为两部分:
\begin{equation}
\begin{aligned}
\sum_i^N\chi_i=\sum_i^{N/2}\psi_i+\sum_i^{N/2}\overline{\psi_i}
\end{aligned}
\end{equation}
对于双求和:
\begin{equation}
\begin{aligned}
\sum_i^N\sum_j^N\chi_i\chi_j&=\sum_i^N\chi_i\sum_j^N\chi_j\\
&=\sum_i^{N/2}(\psi_i+\overline{\psi_i})\sum_j^{N/2}(\psi_j+\overline{\psi_j})\\
&=\sum_i^{N/2}\sum_j^{N/2}\psi_i\psi_j+\psi_i\overline{\psi_j}+\overline{\psi_i}\psi_j+\overline{\psi_i}\overline{\psi_j}
\end{aligned}
\end{equation}
我们将其转化为空间轨道方程。对于单电子积分:
\begin{equation}
\begin{aligned}
\sum_i^N\langle i|h|i\rangle=\sum_i^{N/2}\langle i|h|i\rangle+\sum_i^{N/2}\langle \overline{i}|h|\overline{i}\rangle=2\sum_i^{N/2}\langle\psi_i|h|\psi_i\rangle
\end{aligned}
\end{equation}
双电子积分项:
\begin{equation}
\begin{aligned}
\frac12\sum_i^N&\sum_j^N\langle ij|ij\rangle-\langle ij|ji\rangle\\
&=\frac12\sum_i^{N/2}\sum_j^{N/2}\langle ij|ij\rangle-\langle ij|ji\rangle+\frac12\sum_i^{N/2}\sum_j^{N/2}\langle i\overline{j}|i\overline{j}\rangle-\langle i\overline{j}|\overline{j}i\rangle\\
&+\frac12\sum_i^{N/2}\sum_j^{N/2}\langle \overline{i}j|\overline{i}j\rangle-\langle \overline{i}j|j\overline{i}\rangle+\frac12\sum_i^{N/2}\sum_j^{N/2}\langle \overline{i}\overline{j}|\overline{i}\overline{j}\rangle-\langle \overline{i}\overline{j}|\overline{j}\overline{i}\rangle\\
&=\sum_i^{N/2}\sum_j^{N/2}2\langle\psi_i\psi_j|\psi_i\psi_j\rangle-\langle\psi_i\psi_j|\psi_j\psi_i\rangle
\end{aligned}
\end{equation}
因此,空间轨道下的闭壳层限制性Hartree-Fock能量方程是:
\begin{equation}
\begin{aligned}
E_0=2\sum_i^{N/2}\langle\psi_i|h|\psi_i\rangle+\sum_i^{N/2}\sum_j^{N/2}2\langle\psi_i\psi_j|\psi_i\psi_j\rangle-\langle\psi_i\psi_j|\psi_j\psi_i\rangle
\end{aligned}
\end{equation}

\subsubsection{电子积分}
  这里只介绍HF方法中使用到的电子积分,并不展开其计算的解析式。
\begin{itemize}
  \item 单电子积分
  \begin{itemize}
    \item 重叠积分S:$S_{pq}=\langle\psi_p|\psi_q\rangle$
    \item 动能积分T:$T_{pq}=\langle\psi_p|-\frac{1}{2}\nabla^2|\psi_q\rangle$
    \item 核-电子势能积分V:$V_{pq}=\langle\psi_p|\frac{1}{r_C}|\psi_q\rangle$
	\item 核-哈密顿矩阵$H^{core}$:$H^{core}=T_{pq}+V_{pq}$
  \end{itemize}
  \item 双电子积分
\begin{itemize}
\item I:$I_{pqrs} = \int d\mathbf{r_1} d\mathbf{r_2} \phi_p^*(\mathbf{r_1}) \phi_q(\mathbf{r_1}) \phi_r^*(\mathbf{r_2}) \phi_s(\mathbf{r_2})$
  \item 库伦积分J:$J_{ij}=\langle\psi_i\psi_j|\psi_i\psi_j\rangle$
  \item 交换积分K:$K_{ij}=\langle\psi_i\psi_j|\psi_j\psi_i\rangle$
\end{itemize}
\end{itemize}
对于水分子在STO-3G下,共有10个电子、7个分子轨道,所以单电子积分为7*7的对称阵,双电子积分为7*7*7*7的四维矩阵,显然,双电子积分是自洽场计算中最耗时的部分,此处留个疑问:\textbf{存储四维矩阵显然浪费,且处理四维矩阵更耗时,计算程序采用那些策略优化呢?}
\subsection{Roothan方程}
Roothan方程一般是指限制性闭壳层HF方程,也称为Hartree-Fock-Roothan方程。以他为标题是因为下文的代码实现部分是如此。我们先推导一般性的HF方程。假设读者已经了解线性变分法(应该是结构化学的的知识)。\par
首先,给定一个线性变分尝试波函数:
\begin{equation}
\begin{aligned}
|\Phi\rangle=\sum_{i=0}^Nc_i|\Psi_i\rangle
\end{aligned}
\end{equation}
其需要能量极小化的表达式为:
\begin{equation}
\begin{aligned}
E=\langle\Phi|\mathbf{H}|\Phi\rangle=\sum_{ij}c_i^*c_j\langle\Psi_i|\mathbf{H}|\Psi_j\rangle
\end{aligned}
\end{equation}
注意,尝试波函数需要满足归一化:
\begin{equation}
\begin{aligned}
\langle\Phi\Phi\rangle-1=\sum_{ij}c_i^*c_j\langle\Phi\Phi\rangle-1=0
\end{aligned}
\end{equation}
使用Lagrange不定乘子发,定义方程$L$和Lagrange乘子$\epsilon$:
\begin{equation}
\begin{aligned}
L&=\langle\Phi|\mathbf{H}|\Phi\rangle-\epsilon(\langle\Phi\Phi\rangle-1)\\
 &=\sum_{ij}c_i^*c_j\langle\Psi_i|\mathbf{H}|\Psi_j\rangle-\epsilon\left(\sum_{ij}c_i^*c_j\langle\Phi|\Phi\rangle-1\right)
\end{aligned}
\end{equation}
进行线性变分:
\begin{equation}
\begin{aligned}
\delta L&=\sum_{ij}\delta c_i^*c_j\langle\Psi_i|\mathbf{H}|\Psi_j\rangle-\epsilon\sum_{ij}\delta c_i^*c_j\langle\Phi|\Phi\rangle\\&+\sum_{ij}c_i^*\delta c_j\langle\Psi_i|\mathbf{H}|\Psi_j\rangle-\epsilon\sum_{ij}c_i^*\delta c_j\langle\Phi|\Phi\rangle\\
&=2\sum_i\delta c_i^*\left(\sum_j\langle\Psi_i|\mathbf{H}|\Psi_j\rangle c_j-\epsilon \langle\Psi_i|\Psi_j\rangle c_j\right)=0
\end{aligned}
\end{equation}
我简单将复共轭合并,注意,索引i,j是等价的。\\
我们使用上文定义的$H_{ij}=\langle\Psi_i|\mathbf{H}|\Psi_j\rangle$和重叠积分$S_{ij}=\langle\Psi_i|\Psi_j\rangle$,HF方程可以写为:
\begin{equation}
\begin{aligned}
\sum_jH_{ij}c_j=\epsilon\sum_jS_{ij}c_j
\end{aligned}
\end{equation}
当然,一般写为:
\begin{equation}
\begin{aligned}
\mathbf{Hc}=\epsilon\mathbf{Sc}
\end{aligned}
\end{equation}
聪明的你显然注意到了,虽然我前文没有着墨声明:算符=矩阵,积分=张量,大家在符号中也能体会到了。\\
下面引入基函数,假设有K个已知的基函数(显然是我们上文提到的Gaussian形基函数),那么位置的分子轨道可以表示为基函数的线性展开:
\begin{equation}
\begin{aligned}
\psi_i=\sum_{\mu}^KC_{\mu i}\phi_{\mu}
\end{aligned}
\end{equation}
带入Fock算符,HF方程可写为
\begin{equation}
\begin{aligned}
f\sum_vC_{vi}\phi_v=\epsilon_i\sum_vC_{vi}\phi_v
\end{aligned}
\end{equation}
左乘波函数的复共轭得到:
\begin{equation}
\begin{aligned}
\sum_vC_{vi}\langle\phi_v^*|f|\phi_v\rangle=\epsilon_i\sum_vC_{vi}\langle\phi_v^*|\phi_v\rangle
\end{aligned}
\end{equation}
我们定义Fock矩阵$F_{\mu v}=\langle\phi_v^*|f|\phi_v\rangle$,还有上文提到的重叠积分。那么,Roothan方程可以表示为:
\begin{equation}
\begin{aligned}
\sum_vF_{\mu v}C_{vi}=\epsilon_i\sum_vS_{\mu v}C_{vi}
\end{aligned}
\end{equation}
或者写成矩阵形式:
\begin{equation}
\begin{aligned}
\mathbf{FC}=\mathbf{SC}\epsilon
\end{aligned}
\end{equation}
那么聪明的你,猜猜这些矩阵的形状是什么呢?\par
我们定义总电荷密度:
\begin{equation}
\begin{aligned}
\rho(r)=2\sum_a^{N/2}|\psi_a(r)|^2
\end{aligned}
\end{equation}
将波函数的线性展开带入其中得到:
\begin{equation}
\begin{aligned}
\rho(r)&=2\sum_a^{N/2}\psi_a^*(r)\psi_a(r)\\
&=2\sum_a^{N/2}\sum_vC_{va}^*\phi_v^*(r)\sum_{\mu}C_{\mu a}\phi_{\mu}(r)\\
&=\sum_{\mu v}\left(2\sum_a^{N/2}C_{va}^*C_{\mu a}\right)\phi_v^*(r)\phi_{\mu}(r)\\
&=\sum_{\mu v}P_{\mu v}\phi_v^*(r)\phi_{\mu}(r)
\end{aligned}
\end{equation}
我们得到了一个重要的概念:密度矩阵$P_{\mu v}=2\sum_a^{N/2}C_{va}^*C_{\mu a}$
可以证明Fock算符是波函数的本正算符,结合上节的推导可得到其表达式:
\begin{equation}
\begin{aligned}
f=h+\sum_{a}^{N/2}2J_{a}-K_{a}
\end{aligned}
\end{equation}
将波函数的线性展开带入Fock算符得到,并且写为矩阵形式(主要是好写):
\begin{equation}
\begin{aligned}
F_{\mu v}&=H_{\mu v}^{core}+\sum_a^{N/2}\sum_{ij}C_{ia}C_{ja}^*[2\langle ij|ij\rangle-\langle ij|ji\rangle]\\
&=H_{\mu v}^{core}+\sum_{ij}P_{ij}[\langle ij|ij\rangle-\frac12\langle ij|ji\rangle]\\
\end{aligned}
\end{equation}
我们发现,Fock矩阵只与密度矩阵有关,也就是Fock矩阵依赖于展开系数。展开系数又可以通过求解Roothan方程得到,如此迭代直到能量最低,这就是所谓“自洽(self-consistent)”。

\newpage

\section{程序编写}
\subsection{环境准备}
  对于开发来说,尤其是计算化学程序,Linux绝对是最好的选择,相比Windows下需要考虑的更少,开发效率更高、更灵活。显然大多数电脑系统还是Windows,所以建议大家安装WSL2(\href{https://mp.weixin.qq.com/s/DoK08_Qpvt28ZjAZQa8MZA}{安装教程}),安装好gcc和gfortran,建议使用Anaconda简化python库文件的管理,我们后续代码需要Numpy。其次建议使用MacOS系统。本文所有代码均在WSL2中运行,Python版本为3.10\par
  教程使用Python,当然使用什么语言无所谓,C/C++、Fortran都可以,他们的效率可能更高,但需要一定熟练度,否则不一定快于Numpy(因为其底层还是用C/Fortran写的),使用Python我认为对新手更Friendly。我会提供单、双电子积分,可以直接load,如果想跑其他结构,建议安装Pyscf。\par
对于Python语法,我不会用很复杂,因为只涉及矩阵处理,但也没精力再写一遍Numpy教程。这里假设读者水平为大学上过编程课程,会使用Chat-GPT等AI大模型问问题,基本上课余时间一周肯定能写出来!
\subsection{程序流程}
根据Szabo和Ostlund书中146页描述,自洽场迭代步骤为:
\begin{itemize}
  \item 步骤
  \begin{enumerate}
	\item 得到一个密度矩阵初猜
	\item 根据电子积分和初猜构建Fock矩阵
	\item 对角化Fock矩阵
	\item 选择占据轨道并计算新的密度矩阵
	\item 计算HF能量
	\item 计算误差判断是否收敛
	\begin{itemize}
	  \item 如果不收敛,使用新的密度矩阵继续
	  \item 如果收敛,输出能量
	\end{itemize}
  \end{enumerate}
\end{itemize}
\subsection{代码实现}
我们的任务目标是:根据提供的单、双电子积分,计算$H_2O$分子在$STO-3G$基组下的HF电子能量,参考值:$-84.1513215474753$\\
载入环境
\begin{lstlisting}[style = Python]
import numpy as np
import scipy 
\end{lstlisting}
将.npy二进制文件和python脚本文件放在同一文件夹中,载入单电子积分和双电子积分,同时获取轨道数nao。我们使用一个单位阵作为初猜,对应步骤1.值得注意,这是一种很粗糙的制作初猜的方式,不同的计算化学程序又不同的制作初猜的方法,初猜越准确,自洽场收敛越快。
\begin{lstlisting}[style = Python]
overlap_matrix = np.load("overlap.npy")
H = np.load("core_hamiltonian.npy")
int2e = np.load("int2e.npy")
nao = len(overlap_matrix[0])
assert nao == len(int2e[0])
dm = np.eye(nao)
\end{lstlisting}
下面我们需要些两个函数,用于构建库伦积分\textbf{J}和交换积分\textbf{K},传入函数的是密度矩阵$dm$,对应步骤2.因为双电子积分是四维矩阵,所以应该是四重循环。根据上文的公式,遍历壳层$p, q, r, s$
\begin{lstlisting}[style = Python]
# Calculate the coulomb matrices from density matrix
def get_j(dm):
    J = np.zeros((nao, nao))  # Initialize the Coulomb matrix

    # Loop over all indices of the Coulomb matrix
    for p in range(nao):
        for q in range(nao):
            # Calculate the Coulomb integral for indices (p,q)
            for r in range(nao):
                for s in range(nao):
                    J[p, q] += dm[r, s] * int2e[p, q, r, s]

    return J

# Calculate the exchange  matrices from density matrix
def get_k(dm):
    K = np.zeros((nao, nao))  # Initialize the K matrix

    # Loop over all indices of the K matrix
    for p in range(nao):
        for q in range(nao):
            # Calculate the K integral for indices (p,q)
            for r in range(nao):
                for s in range(nao):
                    K[p, q] += dm[r, s] * int2e[p, r, q, s]

    return K
\end{lstlisting}
我们还需要一个函数用于构建新的密度矩阵$dm$,将Fock矩阵传入函数。根据密度矩阵的公式,我们要遍历所有占据轨道的系数矩阵。对于本体系共10个电子,也就是5个占据轨道和2个空轨道。在这个函数中,我们要先完成步骤3,将Fock矩阵对角化,得到系数矩阵\textbf{C}。我们需要将所有原子轨道基函数正交化,一个可以的方法是对称正交化。例如定义一个矩阵\textbf{A},满足:
$$\mathbf{A}^{\dagger}\mathbf{S}\mathbf{A}=1$$
令$\mathbf{A}=\mathbf{S}^{-1/2}$,于是有:
$$\mathbf{A}^{\dagger}\mathbf{S}\mathbf{A}=\mathbf{S}^{1/2}\mathbf{S}\mathbf{S}^{-1/2}=\mathbf{S}^{-1/2}\mathbf{S}^{1/2}=\mathbf{S}^{0}=1$$
我们借助矩阵\textbf{A}将Roothan方程转换为本征方程(canonical eigenvalue equation),令$\mathbf{C}=\mathbf{A}\mathbf{C}^{\prime}$
$$\mathbf{F}\mathbf{A}\mathbf{C}^{\prime}=\mathbf{S}\mathbf{A}\mathbf{C}^{\prime}\epsilon$$
$$\mathbf{A}^{\dagger}(\mathbf{F}\mathbf{A}\mathbf{C}^{\prime})=\mathbf{A}^{\dagger}(\mathbf{S}\mathbf{A}\mathbf{C}^{\prime})\epsilon$$
$$(\mathbf{A}^{\dagger}\mathbf{F}\mathbf{A})\mathbf{C}^{\prime}=(\mathbf{A}^{\dagger}(\mathbf{S}\mathbf{A})\mathbf{C}^{\prime}\epsilon$$
$$\mathbf{F}^{\prime}\mathbf{C}^{\prime}=\mathbf{C}^{\prime}\epsilon$$
于是我们可以通过将$\mathbf{F}^{\prime}$矩阵对角化得到$\mathbf{C}^{\prime}$,再将其转换回$\mathbf{C}$通过$\mathbf{C}=\mathbf{A}\mathbf{C}^{\prime}$
可以通过Scipy库实现,我们使用了scipy.linalg.fractional\_matrix\_power()函数和np.linalg.eigh()函数
\begin{lstlisting}[style = Python]
# Calculate the density matrix
def get_dm(fock, nocc):
    dm = np.zeros((nao, nao))
    S = overlap_matrix
    A = scipy.linalg.fractional_matrix_power(S, -0.5)
    F_p = A.T @ fock @ A
    eigs, coeffsm = np.linalg.eigh(F_p)

    c_occ = A @ coeffsm
    c_occ = c_occ[:, :nocc]
    for i in range(nocc):
        for p in range(nao):
            for q in range(nao):
                dm[p, q] += c_occ[p, i] * c_occ[q, i]
    return dm
\end{lstlisting}
OK!准备工作已经充足,可以开始自洽场迭代了!对应步骤5、6\\
我们首先要确定收敛限和最大迭代次数。收敛限即最后收敛能量与上一次循环的HF能量变化小于一个值,我们设置为$1.0e-10$,最大迭代次数为40次同时将能量初始化
\begin{lstlisting}[style = Python]
# Maximum SCF iterations
max_iter = 100
E_conv = 1.0e-10
# SCF & Previous Energy
SCF_E = 0.0
E_old = 0.0
\end{lstlisting}
根据步骤6书写循环
\begin{lstlisting}[style = Python]
for scf_iter in range(1, max_iter + 1):
    # GET Fock martix
    F = H + 2 * get_j(dm) - get_k(dm)
    assert F.shape == (nao, nao)

    SCF_E = np.sum(np.multiply((H + F), dm))
    dE = SCF_E - E_old
    print('SCF Iteration %3d: Energy = %4.16f dE = % 1.5E' % (scf_iter, SCF_E, dE))

    if (abs(dE) < E_conv):
        print("SCF convergence! Congrats")
        break
    E_old = SCF_E

    dm = get_dm(F, 5)

assert(np.abs(SCF_E + 84.1513215474753) < 1.0e-10)
\end{lstlisting}
\newpage
\section{后记}
完整的代码我上传在\href{https://github.com/Yxwxwx/Penglab_tutorial}{gitlab}上,从上面可以下载二进制积分文件,npy和源代码。本问的代码非常简单,只有几十行,而且有非常多可以改良的地方,比如\textbf{get\_j}函数中使用了四重循环,这在python中是灾难性的代码;而且也并未考虑积分对称性。我列出几个可以继续思考的地方:
\begin{enumerate}
	\item 结合参考书,实现UHF代码
	\item 使用np.einsum代替循环和一些内置函数以提高效率
	\item 考虑双电子积分的八重对称性加速构建Fock矩阵
	\item 使用DIIS技术加速SCF收敛
	\item 使用Cython,C/C++,Fortran等静态编程语言重写代码以加速
	\item 安装并学习Pyscf,尝试计算更多电子结构
	\item ...
\end{enumerate}\par
能自己写一个计算化学代码并且和自己常用的计算化学软件对应上,其实是一个很有成就感的过程。Keep coding!Keep thinking!



\newpage
\begin{thebibliography}{99}
    \bibitem{a}Szabo and Ostlund. \emph{Modern Quantum Chemistry: Introduction to Advanced Electronic Structure Theory}[M]. New York:Dover Publications,1996.
\end{thebibliography}

\newpage

\begin{appendices}
    \renewcommand{\thesection}{\Alph{section}}
    \section{完整代码(简易版)}
\begin{lstlisting}[style = Python]
import numpy as np
import scipy

overlap_matrix = np.load("overlap.npy")
H = np.load("core_hamiltonian.npy")
int2e = np.load("int2e.npy")
nao = len(overlap_matrix[0])
assert nao == len(int2e[0])
assert int2e.shape == (nao, nao, nao, nao)
dm = np.eye(nao)

# Calculate the coulomb matrices from density matrix
def get_j(dm):
    J = np.zeros((nao, nao))  # Initialize the Coulomb matrix

    # Loop over all indices of the Coulomb matrix
    for p in range(nao):
        for q in range(nao):
            # Calculate the Coulomb integral for indices (p,q)
            for r in range(nao):
                for s in range(nao):
                    J[p, q] += dm[r, s] * int2e[p, q, r, s]

    return J

# Calculate the exchange  matrices from density matrix
def get_k(dm):
    K = np.zeros((nao, nao))  # Initialize the K matrix

    # Loop over all indices of the K matrix
    for p in range(nao):
        for q in range(nao):
            # Calculate the K integral for indices (p,q)
            for r in range(nao):
                for s in range(nao):
                    K[p, q] += dm[r, s] * int2e[p, r, q, s]

    return K

# Calculate the density matrix
def get_dm(fock, nocc):
    dm = np.zeros((nao, nao))
    S = overlap_matrix
    A = scipy.linalg.fractional_matrix_power(S, -0.5)
    F_p = A.T @ fock @ A
    eigs, coeffsm = np.linalg.eigh(F_p)

    c_occ = A @ coeffsm
    c_occ = c_occ[:, :nocc]
    for i in range(nocc):
        for p in range(nao):
            for q in range(nao):
                dm[p, q] += c_occ[p, i] * c_occ[q, i]
    return dm
# Maximum SCF iterations
max_iter = 100
E_conv = 1.0e-10
# SCF & Previous Energy
SCF_E = 0.0
E_old = 0.0
for scf_iter in range(1, max_iter + 1):
    # GET Fock martix
    F = H + 2 * get_j(dm) - get_k(dm)
    assert F.shape == (nao, nao)

    SCF_E = np.sum(np.multiply((H + F), dm))
    dE = SCF_E - E_old
    print('SCF Iteration %3d: Energy = %4.16f dE = % 1.5E' % (scf_iter, SCF_E, dE))

    if (abs(dE) < E_conv):
        print("SCF convergence! Congrats")
        break
    E_old = SCF_E

    dm = get_dm(F, 5)

assert(np.abs(SCF_E + 84.1513215474753) < 1.0e-10)
\end{lstlisting}
\newpage
\section{完整代码(Libcint版)}
\begin{lstlisting}[style = Python]
import numpy as np
import scipy
import ctypes
from pyscf import gto
import time

# slots of atm
CHARGE_OF       = 0
PTR_COORD       = 1
NUC_MOD_OF      = 2
PTR_ZETA        = 3
PTR_FRAC_CHARGE = 3
RESERVE_ATMLOT1 = 4
RESERVE_ATMLOT2 = 5
ATM_SLOTS       = 6


# slots of bas
ATOM_OF         = 0
ANG_OF          = 1
NPRIM_OF        = 2
NCTR_OF         = 3
KAPPA_OF        = 4
PTR_EXP         = 5
PTR_COEFF       = 6
RESERVE_BASLOT  = 7
BAS_SLOTS       = 8

# Create a molecular object for H2O molecule
mol = gto.M(atom='O 0 0 0; H 0 -0.757 0.587; H 0 0.757 0.587', basis='6-31g')

# necessary parameters for Libcint
atm = mol._atm.astype(np.intc)
bas = mol._bas.astype(np.intc)
env = mol._env.astype(np.double)
nao = mol.nao_nr().astype(np.intc)
nshls = len(bas)
natm = len(atm)

_cint = ctypes.cdll.LoadLibrary('/home/yx/cint_and_xc/lib/libcint.so')

def get_ovlp_matrix():
    
    ovlp_matrix = np.zeros((nao, nao), order='F')
    
    _cint.cint1e_ovlp_sph.argtypes = [
    np.ctypeslib.ndpointer(dtype=np.double, ndim=2),
    (ctypes.c_int * 2),
    np.ctypeslib.ndpointer(dtype=np.intc, ndim=2),
    ctypes.c_int,
    np.ctypeslib.ndpointer(dtype=np.intc, ndim=2),
    ctypes.c_int,
    np.ctypeslib.ndpointer(dtype=np.double, ndim=1)
]
    

    _cint.CINTcgto_spheric.restype = ctypes.c_int
    _cint.CINTcgto_spheric.argtypes = [ctypes.c_int, np.ctypeslib.ndpointer(dtype=np.intc, ndim=2)]
    
    for ipr in range(nshls):
        di = _cint.CINTcgto_spheric(ipr, bas)
        x = 0
        for i in range(ipr):
            x += _cint.CINTcgto_spheric(i, bas)

        for jpr in range(nshls):
            dj = _cint.CINTcgto_spheric(jpr, bas)
            y = 0
            for j in range(jpr):
                y += _cint.CINTcgto_spheric(j, bas)

            buf = np.empty((di, dj), order='F')
            _cint.cint1e_ovlp_sph(buf, (ctypes.c_int * 2)(ipr, jpr), atm, natm, bas, nshls, env)

            # Update the overlap matrix with the values from buf
            ovlp_matrix[x: x + di, y : y + dj] = buf

    return ovlp_matrix
def get_core_hamiltonian():
    
    core_h = np.zeros((nao, nao), order='F')
    
    _cint.cint1e_nuc_sph.argtypes = [
    np.ctypeslib.ndpointer(dtype=np.double, ndim=2),
    (ctypes.c_int * 2),
    np.ctypeslib.ndpointer(dtype=np.intc, ndim=2),
    ctypes.c_int,
    np.ctypeslib.ndpointer(dtype=np.intc, ndim=2),
    ctypes.c_int,
    np.ctypeslib.ndpointer(dtype=np.double, ndim=1)
]
    _cint.cint1e_kin_sph.argtypes = [
    np.ctypeslib.ndpointer(dtype=np.double, ndim=2),
    (ctypes.c_int * 2),
    np.ctypeslib.ndpointer(dtype=np.intc, ndim=2),
    ctypes.c_int,
    np.ctypeslib.ndpointer(dtype=np.intc, ndim=2),
    ctypes.c_int,
    np.ctypeslib.ndpointer(dtype=np.double, ndim=1)
]
    

    _cint.CINTcgto_spheric.restype = ctypes.c_int
    _cint.CINTcgto_spheric.argtypes = [ctypes.c_int, np.ctypeslib.ndpointer(dtype=np.intc, ndim=2)]
    
    for ipr in range(nshls):
        di = _cint.CINTcgto_spheric(ipr, bas)
        x = 0
        for i in range(ipr):
            x += _cint.CINTcgto_spheric(i, bas)

        for jpr in range(nshls):
            dj = _cint.CINTcgto_spheric(jpr, bas)
            y = 0
            for j in range(jpr):
                y += _cint.CINTcgto_spheric(j, bas)

            buf1 = np.empty((di, dj), order='F')
            buf2 = np.empty((di, dj), order='F')
            
            _cint.cint1e_nuc_sph(buf1, (ctypes.c_int * 2)(ipr, jpr), atm, natm, bas, nshls, env)
            _cint.cint1e_kin_sph(buf2, (ctypes.c_int * 2)(ipr, jpr), atm, natm, bas, nshls, env)

            # Update the overlap matrix with the values from buf
            core_h[x: x + di, y : y + dj] += buf1
            core_h[x: x + di, y : y + dj] += buf2

    return core_h
def get_int2e():
    int2e = np.zeros((nao, nao, nao, nao), order='F')
    
    _cint.cint2e_sph.argtypes = [
    np.ctypeslib.ndpointer(dtype=np.double, ndim=4),
    (ctypes.c_int * 4),
    np.ctypeslib.ndpointer(dtype=np.intc, ndim=2),
    ctypes.c_int,
    np.ctypeslib.ndpointer(dtype=np.intc, ndim=2),
    ctypes.c_int,
    np.ctypeslib.ndpointer(dtype=np.double, ndim=1),
    ctypes.POINTER(ctypes.c_void_p)
]
    _cint.CINTcgto_spheric.restype = ctypes.c_int
    _cint.CINTcgto_spheric.argtypes = [ctypes.c_int, np.ctypeslib.ndpointer(dtype=np.intc, ndim=2)]
    
    for ipr in range(nshls):
        di = _cint.CINTcgto_spheric(ipr, bas)
        x = 0
        for i in range(ipr):
            x += _cint.CINTcgto_spheric(i, bas)

        for jpr in range(nshls):
            dj = _cint.CINTcgto_spheric(jpr, bas)
            y = 0
            for j in range(jpr):
                y += _cint.CINTcgto_spheric(j, bas)

            for kpr in range(nshls):
                dk = _cint.CINTcgto_spheric(kpr, bas)
                z = 0
                for k in range(kpr):
                    z += _cint.CINTcgto_spheric(k, bas)

                for lpr in range(nshls):
                    dl = _cint.CINTcgto_spheric(lpr, bas)
                    w = 0
                    for l in range(lpr):
                        w += _cint.CINTcgto_spheric(l, bas)

                    buf = np.empty((di, dj, dk, dl), order='F')
                    _cint.cint2e_sph(buf, (ctypes.c_int * 4)(ipr, jpr, kpr, lpr), atm, natm, bas, nshls, env, ctypes.POINTER(ctypes.c_void_p)())

                    # Update the overlap matrix with the values from buf
                    int2e[x: x + di, y : y + dj, z : z + dk, w : w + dl] = buf
        int2e.reshape([nao, nao, nao, nao])
        
    return int2e
def make_j(D):
    return np.einsum('pqrs,rs->pq', I, D, optimize=True) 
def make_k(D):
    return np.einsum('prqs,rs->pq', I, D, optimize=True)  
def make_d(fock, norb):
    eigs, coeffs = scipy.linalg.eigh(fock, S)
    c_occ = coeffs[:, :norb]
    return np.einsum('pi,qi->pq', c_occ, c_occ, optimize=True)

if __name__ == '__main__':
    start = time.time()
    S = get_ovlp_matrix()   
    H = get_core_hamiltonian()
    I = get_int2e()
    # SCF & Previous Energy
    SCF_E = 0.0
    E_old = 0.0
    # start DM
    D = make_d(H, 5)
    # ==> RHF-SCF Iterations <==
    for scf_iter in range(1, 100 + 1):

        # GET Fock martix
        F = H + 2 * make_j(D) - make_k(D)
        '''error vector = FDS - SDF '''
        diis_r = F.dot(D).dot(S) - S.dot(D).dot(F)
        SCF_E = np.einsum('pq,pq->', (H + F), D, optimize=True)
        dE = SCF_E - E_old
        dRMS = 0.5 * np.mean(diis_r ** 2) ** 0.5
        print('SCF Iteration %3d: Energy = %4.16f dE = % 1.5E dRMS = %1.5E' % (scf_iter, SCF_E, dE, dRMS))

        if (abs(dE) < 1e-10) and (dRMS < 1e-10):
            end = time.time()
            print("SCF convergence! Congrats")
            print("Time used: ", end - start)
            break
        E_old = SCF_E
        D = make_d(F, 5)
        
  
\end{lstlisting}
\end{appendices}

\end{document}
